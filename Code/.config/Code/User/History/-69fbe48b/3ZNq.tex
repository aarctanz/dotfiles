\documentclass[12pt,a4paper]{article}
\usepackage[margin=1in]{geometry}
\usepackage{graphicx}
\usepackage{hyperref}
\usepackage{amsmath}
\usepackage{titlesec}

% Title formatting
\titleformat{\section}[block]{\large\bfseries}{\thesection.}{0.5em}{}
\titleformat{\subsection}[block]{\normalsize\bfseries}{\thesubsection.}{0.5em}{}

\title{\textbf{Mess Menu Optimization Project Report}}
% \author{Abhinav Kumar, 524110022}
\date{}

\begin{document}

\maketitle
\vspace{-1em} % Reduces the gap between title and the names
\begin{center}
    Abhinav Kumar (524110022) \\
    Hritik Ladla Patwey (524410015) \\ 
    Praveen Kumar Gupta (524110019) \\ 
    Adarsh Kumar Gupta (524110032) \\ 
\end{center}
% \tableofcontents
% \newpage

\section{Introduction}
The goal of this project is to figure out how to make the mess menu better, more satisfying for students, and at the same time, nutritional. We all know how important the mess menu is to students, so using data to make it better seems like a pretty good idea.

\section{Problem Statement}
We’re tackling a few problems with the current mess setup:
\begin{itemize}
    \item We don’t really have much data on what students actually like or dislike about the menu.
    \item We don’t know if the meals meet the daily nutritional needs (RDA) for students.
    \item Balancing variety, taste, and practicality in the kitchen is tough.
\end{itemize}

\section{Goals and Research Questions}
\subsection{Main Questions}
\begin{enumerate}
    \item Which meals do students like the most and which ones are not so great?
    \item How happy are students with the current menu, and what changes would they suggest?
    \item Do the meals actually provide the right amount of nutrients to meet daily requirements?
\end{enumerate}

\subsection{Secondary Goals}
\begin{itemize}
    \item Suggest menu changes that are both practical and cost-effective.
    \item Close any gaps in the nutrients and make sure the menu is more in line with health standards.
\end{itemize}

\section{Methodology}
\subsection{Data Collection}
\begin{itemize}
    \item We’ll create a survey on Google Forms to gather info on student preferences, satisfaction, and any suggestions for improvement.
    \item We’ll also look at the current menu and portion sizes with help from the mess staff.
\end{itemize}

\subsection{Nutritional Analysis}
\begin{itemize}
    \item We’ll use nutrition databases to figure out how much protein, vitamins, and other important nutrients are in each meal.
    \item These values will be compared to the ICMR Recommended Dietary Allowances (RDA) for male students aged 18 to 25.
\end{itemize}

\subsection{Data Analysis}
\begin{itemize}
    \item We’ll look through the survey data to see patterns and figure out what students really prefer.
    \item We’ll use some statistical tools to measure satisfaction and figure out any major complaints.
    \item We’ll spot any nutritional gaps and suggest practical changes that could help.
\end{itemize}

\section{Timeline}
\begin{tabular}{|l|l|}
\hline
\textbf{Phase} & \textbf{Duration} \\
\hline
Survey Design and Distribution & Weeks 1--2 \\
Data Collection and Cleaning & Weeks 3--5 \\
Data Analysis & Weeks 6--8 \\
Nutritional Analysis & Weeks 9--10 \\
Recommendations and Report Writing & Weeks 11--12 \\
\hline
\end{tabular}

\section{Anticipated Challenges}
\begin{itemize}
    \item \textbf{Low Response Rate:} We’ll need to promote the survey a lot and send out reminders to get a good number of responses.
    \item \textbf{Bias in Responses:} To get honest answers, we’ll make sure the survey is anonymous and encourage everyone to speak up.
    \item \textbf{Nutritional Data Accuracy:} We’ll use trusted databases, but we may also need to ask faculty for help in interpreting some nutritional info.
\end{itemize}

\section{Expected Outcomes}
\begin{itemize}
    \item A clear understanding of what students like and dislike about the menu, and what needs to be improved.
    \item An analysis of how well the current meals meet students’ nutritional needs.
    \item Practical suggestions to improve the menu, while keeping in mind feasibility and costs.
\end{itemize}

\section{Tools and Resources}
\subsection{Resources}
\begin{itemize}
    \item Google Forms to collect the survey data.
    \item Python (with pandas, matplotlib) or Excel for analyzing the data.
    \item Nutritional databases like the Indian Food Composition Tables (IFCT).
\end{itemize}

\subsection{Faculty Guidance}
\begin{itemize}
    \item We’ll get feedback from faculty on how to improve the survey, analyze the data, and come up with solid recommendations.
    \item Faculty will also help us understand nutritional guidelines and how to interpret them.
\end{itemize}

\section{Conclusion}
This project aims to optimize the mess menu using data to improve student satisfaction, meet nutritional standards, and provide actionable suggestions. In the end, we hope our findings make the dining experience better for everyone and help students maintain a healthy lifestyle.

\section{References}
\begin{itemize}
    \item Indian Council of Medical Research (ICMR) Guidelines.
    \item Nutritional Databases and Tools.
    \item Literature related to food preferences and faculty input.
\end{itemize}

\end{document}
