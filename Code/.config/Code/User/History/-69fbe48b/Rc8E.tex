\documentclass[12pt,a4paper]{article}
\usepackage[margin=1in]{geometry}
\usepackage{graphicx}
\usepackage{hyperref}
\usepackage{amsmath}
\usepackage{titlesec}

% Title formatting
\titleformat{\section}[block]{\large\bfseries}{\thesection.}{0.5em}{}
\titleformat{\subsection}[block]{\normalsize\bfseries}{\thesubsection.}{0.5em}{}

\title{\textbf{Mess Menu Optimization Project Report}}
\author{Abhinav Kumar, 524110022}
\date{}

\begin{document}

\maketitle
\begin{center}
    Hritik Ladla Patwey (524410015) \\ 
    Praveen Kumar Gupta (524110019) \\ 
    Adarsh Kumar Gupta (524110032) \\ 
\end{center}
% \tableofcontents
\newpage

\section{Introduction}
The objective of this project is to optimize the mess menu to improve student satisfaction, ensure nutritional adequacy, and align meals with student preferences. The mess menu is a critical aspect of students' daily lives, and data-driven recommendations can enhance their dining experience.

\section{Problem Statement}
This project addresses the following issues:
\begin{itemize}
    \item Limited data on student preferences and satisfaction with the current mess menu.
    \item Uncertainty about whether the nutritional content of meals meets recommended daily allowances (RDA).
    \item Challenges in balancing taste, variety, and operational feasibility.
\end{itemize}

\section{Goals and Research Questions}
\subsection{Primary Questions}
\begin{enumerate}
    \item What are the most and least preferred meals among students?
    \item How satisfied are students with the current mess menu, and what improvements do they suggest?
    \item Are the nutrients in the current meals sufficient to meet daily dietary requirements?
\end{enumerate}

\subsection{Secondary Goals}
\begin{itemize}
    \item Propose practical and cost-effective menu changes.
    \item Address nutrient gaps and align the menu with health standards.
\end{itemize}

\section{Methodology}
\subsection{Data Collection}
\begin{itemize}
    \item A Google Form survey will be designed to gather student preferences, satisfaction levels, and suggestions.
    \item The current menu and portion sizes will be documented in collaboration with mess staff.
\end{itemize}

\subsection{Nutritional Analysis}
\begin{itemize}
    \item Nutritional databases will be used to calculate nutrient content (e.g., protein, calcium, vitamins).
    \item These values will be compared with the ICMR Recommended Dietary Allowances (RDA) for males aged 18--25.
\end{itemize}

\subsection{Data Analysis}
\begin{itemize}
    \item Survey responses will be analyzed to identify patterns and preferences.
    \item Statistical tools will be used to quantify satisfaction levels and grievances.
    \item Nutritional gaps will be highlighted, and actionable adjustments will be recommended.
\end{itemize}

\section{Timeline}
\begin{tabular}{|l|l|}
\hline
\textbf{Phase} & \textbf{Duration} \\
\hline
Survey Design and Distribution & Weeks 1--2 \\
Data Collection and Cleaning & Weeks 3--5 \\
Data Analysis & Weeks 6--8 \\
Nutritional Analysis & Weeks 9--10 \\
Recommendations and Report Writing & Weeks 11--12 \\
\hline
\end{tabular}

\section{Anticipated Challenges}
\begin{itemize}
    \item \textbf{Low Response Rate:} Promote the survey actively and send reminders.
    \item \textbf{Bias in Responses:} Ensure anonymity and encourage honest feedback.
    \item \textbf{Nutritional Data Accuracy:} Use reliable databases and seek faculty guidance.
\end{itemize}

\section{Expected Outcomes}
\begin{itemize}
    \item A detailed understanding of student meal preferences and satisfaction levels.
    \item An analysis of the nutritional adequacy of current meals.
    \item Practical recommendations to enhance the menu while maintaining feasibility.
\end{itemize}

\section{Tools and Resources}
\subsection{Resources}
\begin{itemize}
    \item Google Forms for data collection.
    \item Python (pandas, matplotlib) or Excel for data analysis.
    \item Nutritional databases such as Indian Food Composition Tables (IFCT).
\end{itemize}

\subsection{Faculty Guidance}
\begin{itemize}
    \item Feedback on survey design, data analysis, and recommendations.
    \item Assistance in interpreting nutritional guidelines.
\end{itemize}

\section{Conclusion}
This project aims to optimize the mess menu by leveraging data analytics to address student preferences, ensure nutritional adequacy, and provide practical recommendations. The findings will enhance the mess experience and contribute to the well-being of students.

\section{References}
\begin{itemize}
    \item Indian Council of Medical Research (ICMR) Guidelines.
    \item Nutritional Databases and Tools.
    \item Relevant literature and faculty inputs.
\end{itemize}

\end{document}
